% Options for packages loaded elsewhere
\PassOptionsToPackage{unicode}{hyperref}
\PassOptionsToPackage{hyphens}{url}
\PassOptionsToPackage{dvipsnames,svgnames,x11names}{xcolor}
%
\documentclass[
]{scrartcl}

\usepackage{amsmath,amssymb}
\usepackage{iftex}
\ifPDFTeX
  \usepackage[T1]{fontenc}
  \usepackage[utf8]{inputenc}
  \usepackage{textcomp} % provide euro and other symbols
\else % if luatex or xetex
  \usepackage{unicode-math}
  \defaultfontfeatures{Scale=MatchLowercase}
  \defaultfontfeatures[\rmfamily]{Ligatures=TeX,Scale=1}
\fi
\usepackage{lmodern}
\ifPDFTeX\else  
    % xetex/luatex font selection
\fi
% Use upquote if available, for straight quotes in verbatim environments
\IfFileExists{upquote.sty}{\usepackage{upquote}}{}
\IfFileExists{microtype.sty}{% use microtype if available
  \usepackage[]{microtype}
  \UseMicrotypeSet[protrusion]{basicmath} % disable protrusion for tt fonts
}{}
\makeatletter
\@ifundefined{KOMAClassName}{% if non-KOMA class
  \IfFileExists{parskip.sty}{%
    \usepackage{parskip}
  }{% else
    \setlength{\parindent}{0pt}
    \setlength{\parskip}{6pt plus 2pt minus 1pt}}
}{% if KOMA class
  \KOMAoptions{parskip=half}}
\makeatother
\usepackage{xcolor}
\setlength{\emergencystretch}{3em} % prevent overfull lines
\setcounter{secnumdepth}{3}
% Make \paragraph and \subparagraph free-standing
\ifx\paragraph\undefined\else
  \let\oldparagraph\paragraph
  \renewcommand{\paragraph}[1]{\oldparagraph{#1}\mbox{}}
\fi
\ifx\subparagraph\undefined\else
  \let\oldsubparagraph\subparagraph
  \renewcommand{\subparagraph}[1]{\oldsubparagraph{#1}\mbox{}}
\fi

\providecommand{\tightlist}{%
  \setlength{\itemsep}{0pt}\setlength{\parskip}{0pt}}\usepackage{longtable,booktabs,array}
\usepackage{calc} % for calculating minipage widths
% Correct order of tables after \paragraph or \subparagraph
\usepackage{etoolbox}
\makeatletter
\patchcmd\longtable{\par}{\if@noskipsec\mbox{}\fi\par}{}{}
\makeatother
% Allow footnotes in longtable head/foot
\IfFileExists{footnotehyper.sty}{\usepackage{footnotehyper}}{\usepackage{footnote}}
\makesavenoteenv{longtable}
\usepackage{graphicx}
\makeatletter
\def\maxwidth{\ifdim\Gin@nat@width>\linewidth\linewidth\else\Gin@nat@width\fi}
\def\maxheight{\ifdim\Gin@nat@height>\textheight\textheight\else\Gin@nat@height\fi}
\makeatother
% Scale images if necessary, so that they will not overflow the page
% margins by default, and it is still possible to overwrite the defaults
% using explicit options in \includegraphics[width, height, ...]{}
\setkeys{Gin}{width=\maxwidth,height=\maxheight,keepaspectratio}
% Set default figure placement to htbp
\makeatletter
\def\fps@figure{htbp}
\makeatother
\newlength{\cslhangindent}
\setlength{\cslhangindent}{1.5em}
\newlength{\csllabelwidth}
\setlength{\csllabelwidth}{3em}
\newlength{\cslentryspacingunit} % times entry-spacing
\setlength{\cslentryspacingunit}{\parskip}
\newenvironment{CSLReferences}[2] % #1 hanging-ident, #2 entry spacing
 {% don't indent paragraphs
  \setlength{\parindent}{0pt}
  % turn on hanging indent if param 1 is 1
  \ifodd #1
  \let\oldpar\par
  \def\par{\hangindent=\cslhangindent\oldpar}
  \fi
  % set entry spacing
  \setlength{\parskip}{#2\cslentryspacingunit}
 }%
 {}
\usepackage{calc}
\newcommand{\CSLBlock}[1]{#1\hfill\break}
\newcommand{\CSLLeftMargin}[1]{\parbox[t]{\csllabelwidth}{#1}}
\newcommand{\CSLRightInline}[1]{\parbox[t]{\linewidth - \csllabelwidth}{#1}\break}
\newcommand{\CSLIndent}[1]{\hspace{\cslhangindent}#1}

\usepackage{scrlayer-scrpage}
\usepackage{lastpage}
\lohead{ATLS vs Standard Care Trial SAP}
\rohead{ClinicalTrials.gov ID NCT06321419}
\cfoot{\thepage\ of \pageref{LastPage} }
\makeatletter
\makeatother
\makeatletter
\makeatother
\makeatletter
\@ifpackageloaded{caption}{}{\usepackage{caption}}
\AtBeginDocument{%
\ifdefined\contentsname
  \renewcommand*\contentsname{Table of contents}
\else
  \newcommand\contentsname{Table of contents}
\fi
\ifdefined\listfigurename
  \renewcommand*\listfigurename{List of Figures}
\else
  \newcommand\listfigurename{List of Figures}
\fi
\ifdefined\listtablename
  \renewcommand*\listtablename{List of Tables}
\else
  \newcommand\listtablename{List of Tables}
\fi
\ifdefined\figurename
  \renewcommand*\figurename{Figure}
\else
  \newcommand\figurename{Figure}
\fi
\ifdefined\tablename
  \renewcommand*\tablename{Table}
\else
  \newcommand\tablename{Table}
\fi
}
\@ifpackageloaded{float}{}{\usepackage{float}}
\floatstyle{ruled}
\@ifundefined{c@chapter}{\newfloat{codelisting}{h}{lop}}{\newfloat{codelisting}{h}{lop}[chapter]}
\floatname{codelisting}{Listing}
\newcommand*\listoflistings{\listof{codelisting}{List of Listings}}
\makeatother
\makeatletter
\@ifpackageloaded{caption}{}{\usepackage{caption}}
\@ifpackageloaded{subcaption}{}{\usepackage{subcaption}}
\makeatother
\makeatletter
\@ifpackageloaded{tcolorbox}{}{\usepackage[skins,breakable]{tcolorbox}}
\makeatother
\makeatletter
\@ifundefined{shadecolor}{\definecolor{shadecolor}{rgb}{.97, .97, .97}}
\makeatother
\makeatletter
\makeatother
\makeatletter
\makeatother

\usepackage{hyphenat}
\usepackage{ifthen}
\usepackage{calc}
\usepackage{calculator}



\usepackage{graphicx}
\usepackage{geometry}
\usepackage{afterpage}
\usepackage{tikz}
\usetikzlibrary{calc}
\usetikzlibrary{fadings}
\usepackage[pagecolor=none]{pagecolor}


% Set the titlepage font families







% Set the coverpage font families

\ifLuaTeX
  \usepackage{selnolig}  % disable illegal ligatures
\fi
\IfFileExists{bookmark.sty}{\usepackage{bookmark}}{\usepackage{hyperref}}
\IfFileExists{xurl.sty}{\usepackage{xurl}}{} % add URL line breaks if available
\urlstyle{same} % disable monospaced font for URLs
\hypersetup{
  pdftitle={Effects of Advanced Trauma Life Support® Training Compared to Standard Care on Adult Trauma Patient Outcomes: A Cluster Randomised Trial},
  colorlinks=true,
  linkcolor={blue},
  filecolor={Maroon},
  citecolor={Blue},
  urlcolor={Blue},
  pdfcreator={LaTeX via pandoc}}

\title{Effects of Advanced Trauma Life Support\textsuperscript{®}
Training Compared to Standard Care on Adult Trauma Patient Outcomes: A
Cluster Randomised Trial}
\usepackage{etoolbox}
\makeatletter
\providecommand{\subtitle}[1]{% add subtitle to \maketitle
  \apptocmd{\@title}{\par {\large #1 \par}}{}{}
}
\makeatother
\subtitle{Statistical Analysis Plan\\
Version 0.1.0, 2024-05-13}
\author{}
\date{}

\begin{document}
%%%%% begin titlepage extension code


\begin{titlepage}

%%% TITLE PAGE START

% Set up alignment commands
%Page
\newcommand{\titlepagepagealign}{
\ifthenelse{\equal{center}{right}}{\raggedleft}{}
\ifthenelse{\equal{center}{center}}{\centering}{}
\ifthenelse{\equal{center}{left}}{\raggedright}{}
}


\newcommand{\titleandsubtitle}{
% Title and subtitle
{\fontsize{20}{24.0}\selectfont
{\uppercase{\nohyphens{Effects of Advanced Trauma Life
Support\textsuperscript{®} Training Compared to Standard Care on Adult
Trauma Patient Outcomes: A Cluster Randomised Trial}}}\par
}%

\vspace{\betweentitlesubtitle}
{
{\Large{\nohyphens{Statistical Analysis Plan\\
Version 0.1.0, 2024-05-13}}}\par
}}
\newcommand{\titlepagetitleblock}{
\rule{\textwidth}{0.4pt} % Thin horizontal rule
\vspace{0.025\textheight} % Whitespace between the top rules and title

\titleandsubtitle

\vspace{0.025\textheight} 
\rule{0.3\textwidth}{0.4pt} % Short horizontal rule under the title
}
\newcommand{\authorstyle}[1]{{\Large{#1}}}

\newcommand{\affiliationstyle}[1]{{\large{#1}}}

\newcommand{\titlepageauthorblock}{
{\authorstyle{\\}}
}

\newcommand{\titlepageaffiliationblock}{
\hangindent=1em
\hangafter=1
{\affiliationstyle{


\vspace{1\baselineskip} 
}}
}
\newcommand{\headerstyled}{%
{}
}
\newcommand{\footerstyled}{%
{\large{\textsc{}}}
}
\newcommand{\datestyled}{%
{}
}


\newcommand{\titlepageheaderblock}{\headerstyled}

\newcommand{\titlepagefooterblock}{
\footerstyled
}

\newcommand{\titlepagedateblock}{
\datestyled
}

%set up blocks so user can specify order
\newcommand{\titleblock}{\newlength{\betweentitlesubtitle}
\setlength{\betweentitlesubtitle}{\baselineskip}
{

{\titlepagetitleblock}
}

\vspace{0.1\textheight}
}

\newcommand{\authorblock}{}

\newcommand{\affiliationblock}{}

\newcommand{\logoblock}{}

\newcommand{\footerblock}{}

\newcommand{\dateblock}{}

\newcommand{\headerblock}{}

\thispagestyle{empty} % no page numbers on titlepages


\newlength{\minipagewidth}
\setlength{\minipagewidth}{\textwidth}
\raggedright % single minipage
% [position of box][box height][inner position]{width}
% [s] means stretch out vertically; assuming there is a vfill
\begin{minipage}[b][\textheight][s]{\minipagewidth}
\titlepagepagealign
\titleblock

\authorblock

\vfill

\logoblock

\footerblock
\par

\end{minipage}\ifthenelse{\equal{}{right} \OR \equal{}{leftright} }{
\hspace{\B}
\vrulecode}{}
\clearpage
%%% TITLE PAGE END
\end{titlepage}
\setcounter{page}{1}

%%%%% end titlepage extension code
\ifdefined\Shaded\renewenvironment{Shaded}{\begin{tcolorbox}[interior hidden, borderline west={3pt}{0pt}{shadecolor}, sharp corners, enhanced, breakable, boxrule=0pt, frame hidden]}{\end{tcolorbox}}\fi

\renewcommand*\contentsname{Table of contents}
{
\hypersetup{linkcolor=}
\setcounter{tocdepth}{3}
\tableofcontents
}
\newpage{}

\hypertarget{administrative-information}{%
\section{Administrative information}\label{administrative-information}}

\hypertarget{study-identifiers}{%
\subsection{Study identifiers}\label{study-identifiers}}

\begin{itemize}
\tightlist
\item
  Protocol version 1.1.0 dated 2024-05-09
\item
  ClinicalTrials.gov ID NCT06321419
\item
  Clinical Trial Registry - India
\end{itemize}

\hypertarget{changelog}{%
\subsection{Changelog}\label{changelog}}

Once version 1.0.0 is finalised, this section will be updated with a
changelog.

\hypertarget{contributors}{%
\subsection{Contributors}\label{contributors}}

\begin{longtable}[]{@{}
  >{\raggedright\arraybackslash}p{(\columnwidth - 4\tabcolsep) * \real{0.4722}}
  >{\raggedright\arraybackslash}p{(\columnwidth - 4\tabcolsep) * \real{0.3611}}
  >{\raggedright\arraybackslash}p{(\columnwidth - 4\tabcolsep) * \real{0.1667}}@{}}
\toprule\noalign{}
\begin{minipage}[b]{\linewidth}\raggedright
Name and ORCID
\end{minipage} & \begin{minipage}[b]{\linewidth}\raggedright
Affiliation
\end{minipage} & \begin{minipage}[b]{\linewidth}\raggedright
Role
\end{minipage} \\
\midrule\noalign{}
\endhead
\bottomrule\noalign{}
\endlastfoot
Martin Gerdin Wärnberg
\href{https://orcid.org/0000-0001-6069-4794}{\includegraphics[width=0.16667in,height=0.16667in]{statistical-analysis-plan_files/mediabag/orcid_16x16.png}}
& Karolinska Institutet & Principal Investigator \\
\end{longtable}

\newpage{}

\hypertarget{trial-synopsis}{%
\section{Trial synopsis}\label{trial-synopsis}}

\textbf{Title} Effects of Advanced Trauma Life
Support\textsuperscript{®} Training Compared to Standard Care on Adult
Trauma Patient Outcomes: A Cluster Randomised Trial

\textbf{Rationale} Trauma is a massive global health issue. Many
training programmes have been developed to help physicians in the
initial management of trauma patients. Among these programmes, Advanced
Trauma Life Support\textsuperscript{®} (ATLS\textsuperscript{®}) is the
most popular, having trained over one million physicians worldwide.
Despite its widespread use, there are no controlled trials showing that
ATLS\textsuperscript{®} improves patient outcomes. Multiple systematic
reviews emphasise the need for such trials.

\textbf{Aim} To compare the effects of ATLS\textsuperscript{®} training
with standard care on outcomes in adult trauma patients.

\textbf{Primary Outcome} In-hospital mortality within 30 days of arrival
at the emergency department.

\textbf{Trial Design} Batched stepped-wedge cluster randomised trial in
India.

\textbf{Trial Population} Adult trauma patients presenting to the
emergency department of a participating hospital.

\textbf{Sample Size} 30 clusters and 4320 patients.

\textbf{Eligibility Criteria}

\emph{Hospitals} are secondary or tertiary hospitals in India that admit
or refer/transfer for admission at least 400 patients with trauma per
year.

\emph{Clusters} are one or more units of physicians providing initial
trauma care in the emergency department of tertiary hospitals in India.

\emph{Patients participants} are adult trauma patients who presents to
the emergency department of participating hospitals and are admitted or
transferred for admission.

\textbf{Intervention} The intervention will be ATLS\textsuperscript{®}
training, a proprietary 2.5 day course teaching a standardised approach
to trauma patient care using the concepts of a primary and secondary
survey. Physicians will be trained in an accredited
ATLS\textsuperscript{®} training facility in India.

\textbf{Ethical Considerations} We will use an opt-out consent approach
for collection of routinely recorded data. We will obtain informed
consent for collection of non-routinely recorded data, such as quality
of life and disability outcomes. Patients who are unconscious or lack a
legally authorized representative will be included under a waiver of
informed consent. Note that consent here refers to consent to data
collection.

\textbf{Trial Period} October 1, 2024, to September 30, 2029

\newpage{}

\hypertarget{special-considerations}{%
\section{Special considerations}\label{special-considerations}}

\hypertarget{funding}{%
\subsection{Funding}\label{funding}}

This trial is not yet fully funded. The Trial Management Group has
decided to proceed with the trial with the expectation that additional
funding will be secured. The Trial Steering Committee will be informed
of the funding status at each meeting. If funding is not secured, the
trial will be stopped. This will likely result in an underpowered trial.
The justification for this decision is that the intervention is
considered standard of care in many countries and the data collection is
considered minimal risk. There is therefore a very small risk of harm to
patient participants, but a potential direct benefit to those patient
participants who receive the intervention. The benefit-risk ratio is
therefore considered to be favourable, even in the case of an
underpowered trial.

\hypertarget{potential-amendments}{%
\subsection{Potential amendments}\label{potential-amendments}}

There are ongoing discussions about re-framing the trial as a hybrid
effectiveness-implementation trial and include a cost-effectiveness
analysis. This would involve adding additional data collection to assess
the implementation and costs of the intervention. This would involve
additional funding and amended ethical approvals.

\newpage{}

\hypertarget{statistical-analysis}{%
\section{Statistical analysis}\label{statistical-analysis}}

\hypertarget{design}{%
\subsection{Design}\label{design}}

This is a batched stepped-wedge cluster randomised trial, composed of 6
batches of identical 12-period 5-sequence design, with one cluster being
assigned to each sequence of each batch\textsuperscript{1}. Each period
is one month, and each cluster will be in the trial for a total of 13
months. The intervention will be implemented during a one-month
transition period, which will be excluded from the analysis. There will
be an overlap of 6 months between successive batches.

\hypertarget{statistical-hypotheses}{%
\subsection{Statistical hypotheses}\label{statistical-hypotheses}}

Our primary statistical hypotheses are:

\begin{itemize}
\tightlist
\item
  \textbf{Null hypothesis}: There is no difference in the primary
  outcome of 30-day in-hospital mortality between those randomised to
  ATLS\textsuperscript{®} and standard care, meaning that the odds ratio
  (OR) for ATLS\textsuperscript{®} vs standard care would be 1.
\item
  \textbf{Alternative hypothesis}: There is a difference in the primary
  outcome of 30-day in-hospital mortality between those randomised to
  ATLS\textsuperscript{®} and standard care, meaning that the OR for
  ATLS\textsuperscript{®} vs standard care would be different from 1.
  Our expectation, based on our pilot study and review of the
  literature, is that the OR will be less than 1, indicating lower odds
  of 30-day in-hospital mortality among those randomised to
  ATLS\textsuperscript{®} group compared to those randomised to the
  standard care group.
\end{itemize}

\hypertarget{statistical-principles}{%
\subsection{Statistical principles}\label{statistical-principles}}

\hypertarget{statistical-software}{%
\subsubsection{Statistical software}\label{statistical-software}}

We will use the R Statistical Software for all
analyses\textsuperscript{2}.

\hypertarget{levels-of-statistical-significance-and-confidence}{%
\subsubsection{Levels of statistical significance and
confidence}\label{levels-of-statistical-significance-and-confidence}}

We will not perform any formal hypothesis testing as part of our planned
interim analyses. We will use a two-sided significance level of 0.05 for
all analyses, and we will report 95\% confidence intervals (CI) for all
estimates. We will not adjust for multiple testing because no secondary
outcome is regarded as singularly more important.

\hypertarget{analysis-populations}{%
\subsection{Analysis populations}\label{analysis-populations}}

The unit of randomisation is the hospital, but the unit of analysis is
the individual patient. The group allocation for a patient depends on
the period in which the patient was admitted to the hospital, and
patients will be considered exposed to the intervention if they were
admitted to the hospital at any time point following the transition
period. We will use an intention-to-treat approach for all analyses. We
will use a CONSORT diagram to display the flow of hospitals, clusters
and patients through the trial. We will report the study according to
the CONSORT guidelines for stepped-wedge randomised
trials\textsuperscript{3}.

\hypertarget{baseline-analyses}{%
\subsection{Baseline analyses}\label{baseline-analyses}}

\hypertarget{cluster-characteristics}{%
\subsubsection{Cluster characteristics}\label{cluster-characteristics}}

We will describe cluster characteristics including location and size
using frequencies and percentages for discrete variables and means,
standard deviations, medians and interquartile ranges (Q1-Q3) for
continuous variables.

\hypertarget{patient-characteristics}{%
\subsubsection{Patient characteristics}\label{patient-characteristics}}

We will describe patient characteristics at baseline per treatment group
and overall using frequencies and percentages for discrete variables and
means, standard deviations, medians and interquartile ranges (Q1-Q3) for
continuous variables. We will not adjust for clustering when presenting
baseline characteristics.

\hypertarget{analysis-of-the-primary-outcome}{%
\subsection{Analysis of the primary
outcome}\label{analysis-of-the-primary-outcome}}

The primary outcomes is in-hospital mortality within 30 days of arrival
at the emergency department and will be analysed as a dichotomous
variable. We will estimate the primary intervention effect as the OR of
death between the ATLS\textsuperscript{®} and standard care arms, with
an OR \textless{} 1 indicating lower odds of death in the
ATLS\textsuperscript{®} arm compared to the standard care arm and vice
versa.

\hypertarget{main-analysis-mixed-effects-binomial-model-with-logit-link}{%
\subsubsection{Main analysis: mixed effects binomial model with logit
link}\label{main-analysis-mixed-effects-binomial-model-with-logit-link}}

We will use a mixed effects binomial model with a logit link to estimate
the OR. We will include fixed effects for period and a fixed effect for
intervention exposure. The primary analysis will allow for clustering by
as a random cluster and random cluster by period effect. The full model
is specified in Equation~\ref{eq-logit-model}. To correct the potential
inflation of the type I error rate due to small number of clusters, the
Kenward and Roger small sample correction will be
used\textsuperscript{4}. This model will be fitted using residual
pseudo-likelihood estimation based on linearization with
subject-specific expansion (RSPL).

\begin{equation}\protect\hypertarget{eq-logit-model}{}{
\text{logit}(\text{Pr}(Y_{bkti} = 1)) = \mu + \beta_{bt} + \theta X_{bkt} + \alpha_{bk} + \gamma_{bkt} 
}\label{eq-logit-model}\end{equation}

Where:

\begin{itemize}
\tightlist
\item
  \(\mu\) is the intercept, representing the baseline log-odds of the
  outcome when all predictors are zero.
\item
  \(\text{Pr}(Y_{bkti} = 1)\) is the probability of death for patient
  \(i = 1, \dotsc,m\) in cluster \(k = 1, \dotsc, 30\) in period
  \(t = 1, \dotsc, 12\) in batch \(b = 1, \dotsc, 6\).
\item
  \(\beta_{bt}\) is the fixed effect of period \(t\) in batch \(b\),
  i.e.~there is a separate period effect for each batch, so that there
  is a total of 72 period effects.
\item
  \(\theta\) is the fixed effect of intervention exposure, i.e.~the
  effect of ATLS\textsuperscript{®} exposure on the probability of
  death.
\item
  \(X_{bkt}\) is the treatment arm for patient \(i\) in cluster \(k\) in
  period \(t\), with \(X_{bkt} = 1\) for ATLS\textsuperscript{®} and
  \(X_{bkt} = 0\) for standard care.
\item
  \(\alpha_{bk}\) is the random effect of cluster \(k\) in batch \(b\),
  i.e.~the random effect of cluster.
\item
  \(\gamma_{bkt}\) is the random effect of cluster \(k\) in period \(t\)
  in batch \(b\), i.e.~the random effect of cluster by period.
\end{itemize}

We will present the effect of ATLS\textsuperscript{®} exposure as an OR
of mortality with an associated 95\% CI, using the standard care arm as
the reference. We will also present the risk difference with a 95\% CI.
We will balance the randomization within each batch on cluster size,
defined as expected monthly volume of eligible patient participants, and
will therefore not adjust the main analysis for cluster size.

\hypertarget{sensitivity-analyses}{%
\subsubsection{Sensitivity analyses}\label{sensitivity-analyses}}

The sensitivity analyses will be conducted to assess the robustness of
the main analysis results to different model specifications. We will
first model the primary outcome using an identity link function to
estimate the risk difference instead of the OR. Henceforth, each
additional sensitivity analyses will be operationalised using two
separate models, one with the logit link and one with the identity link.
We will first explore more complex correlation structures. We will then
model time using a spline function. Finally, we will conduct a fully
adjusted covariate analysis.

\hypertarget{model-with-identify-link}{%
\paragraph{Model with identify link}\label{model-with-identify-link}}

We will use an identity link used to estimate the risk difference,
meaning that the coefficient will be interpreted as the difference in
the probability of death between the ATLS\textsuperscript{®} and
standard care arms. We will present the risk difference with a 95\% CI.
This model is specified in Equation~\ref{eq-identity-model} and will
also be fitted using RSPL. If the binomial model with the identity link
does not converge then only a odds ratio will be reported.

\begin{equation}\protect\hypertarget{eq-identity-model}{}{
\text{Pr}(Y_{bkti} = 1) = \mu + \beta_{bt} + \theta X_{bkt} + \alpha_{bk} + \gamma_{bkt} 
}\label{eq-identity-model}\end{equation}

Where:

\begin{itemize}
\tightlist
\item
  \(\mu\) is the intercept, representing the baseline probability of the
  outcome when all predictors are zero.
\end{itemize}

\hypertarget{models-with-different-correlation-structure}{%
\paragraph{Models with different correlation
structure}\label{models-with-different-correlation-structure}}

We will explore if models with more complicated correlation structures
are a better fit to the data. These models are not being used as our
primary analysis models as there is limited understanding as to when
such models will converge and how to choose between the various
different correlation structures which might be plausible. First, we
will include a discrete time decay correlation structure including a
random cluster effect with auto-regressive structure (AR(1)), described
in Equation~\ref{eq-ar1}.

\begin{equation}\protect\hypertarget{eq-ar1}{}{
\alpha_{bk, t} = \rho \alpha_{bk, t-1} + \epsilon_{t}, \quad \epsilon_{t} \sim N(0, \sigma^2_\alpha)
}\label{eq-ar1}\end{equation}

Where:

\begin{itemize}
\tightlist
\item
  \(\rho\) is the correlation between the random effects of two
  consecutive periods, the period \(t\) and the period \(t-1\).
\item
  \(\alpha_{bk, t}\) is the random effect of cluster \(k\) in period
  \(t\) in batch \(b\).
\item
  \(\epsilon_{t}\) is the error term for period \(t\), which is assumed
  to be normally distributed with mean 0 and variance
  \(\sigma^2_\alpha\).
\end{itemize}

To allow for the randomisation by batches, we will also include a
different secular trend for each batch as a random effect interaction
term between batch and period. The full model is specified in
Equation~\ref{eq-logit-ar1}.

\begin{equation}\protect\hypertarget{eq-logit-ar1}{}{
g(\text{Pr}(Y_{bkti} = 1)) = \mu + \beta_{bt} + \theta X_{bkt} + \alpha_{bk,t} + \gamma_{bkt} + \delta_{bt}
}\label{eq-logit-ar1}\end{equation}

Where:

\begin{itemize}
\tightlist
\item
  \(g(\cdot)\) is the link function.
\item
  \(\alpha_{bk,t}\) is the is the updated random effect of cluster \(k\)
  in batch \(b\) in period \(t\) with the AR(1) correlation structure.
\item
  \(\delta_{bt}\) is the random effect of batch \(b\) in period \(t\).
\end{itemize}

\hypertarget{models-with-random-cluster-by-intervention-effects}{%
\paragraph{Models with random cluster by intervention
effects}\label{models-with-random-cluster-by-intervention-effects}}

Models will also be extended to include random cluster by intervention
effects (with a non-zero covariance term) to examine if results are
sensitive to the assumption of no intervention by cluster interaction.
The model is specified in
Equation~\ref{eq-logit-random-cluster-intervention}.

\begin{equation}\protect\hypertarget{eq-logit-random-cluster-intervention}{}{
g(\text{Pr}(Y_{bkti} = 1)) = \mu + \beta_{bt} + \theta X_{bkt} + \alpha_{bk} + \gamma_{bkt} + u_{bk} \times X_{bkt}
}\label{eq-logit-random-cluster-intervention}\end{equation}

Where:

\begin{itemize}
\tightlist
\item
  \(u_{bk}\) is the random effect of cluster \(k\) by intervention
  interaction.
\end{itemize}

\hypertarget{models-with-time-modelled-with-a-spline-function}{%
\paragraph{Models with time modelled with a spline
function}\label{models-with-time-modelled-with-a-spline-function}}

We will further explore the potential for a time-varying treatment
effect\textsuperscript{5}. To explore if the fixed period effect is both
parsimonious and adequate to represent the extent of any underlying
secular trend, we will model the time effect using natural cubic splines
with knots at the equally spaced time points 3, 6 and 9. This will
result in five spline basis functions, because the natural cubic splines
are modelled with three degrees of freedom but are constrained to be
linear before the first and after the last knot. The model is specified
in Equation~\ref{eq-spline-model}.

\begin{equation}\protect\hypertarget{eq-spline-model}{}{
g(\text{Pr}(Y_{bkti} = 1)) = \mu + \sum_{j=1}^5 \beta_j S_j(t, \{3, 6, 9\}) + \theta X_{bkt} + \alpha_{bk} + \gamma_{bkt}
}\label{eq-spline-model}\end{equation}

Where:

\begin{itemize}
\tightlist
\item
  \(S_j(t, \{3, 6, 9\})\) is the natural cubic spline basis functions
  with knots placed at times 3, 6 and 9.
\item
  \(\beta_j\) is the coefficient for the \(j\)-th spline basis function.
\end{itemize}

\hypertarget{models-exploring-lag-and-weaning-effects}{%
\paragraph{Models exploring lag and weaning
effects}\label{models-exploring-lag-and-weaning-effects}}

Models will also be extended to include an interaction between treatment
and number of periods since first treated, to examine if there is any
indication of a relationship between duration of exposure to the
intervention and outcomes. This will allow us to model different lag
effects (whereby it takes time for the intervention to become embedded
within the culture before its impact can properly start to be realised);
as well as weaning effects (whereby the effect of the intervention
starts to decrease -- or fade). This type of analysis attempts to
disentangle how some clusters end up having a long exposure to the
intervention and others have a much shorter exposure time. The model is
specified in Equation~\ref{eq-lag-weaning-model}.

\begin{equation}\protect\hypertarget{eq-lag-weaning-model}{}{
g(\text{Pr}(Y_{bkti} = 1)) = \mu + \beta_{bt} + \theta X_{bkt} + \theta_{\text{int}} X_{bkt} \times T_{bkt} + \alpha_{bk} + \gamma_{bkt}
}\label{eq-lag-weaning-model}\end{equation}

Where:

\begin{itemize}
\tightlist
\item
  \(\theta_{\text{int}}\) is the coefficient for the interaction between
  treatment and time since first treated.
\item
  \(T_{bkt}\) is the number of periods since first treated.
\end{itemize}

\hypertarget{adjusted-analyses}{%
\subsubsection{Adjusted analyses}\label{adjusted-analyses}}

Fully adjusted covariate analysis will additionally adjust for:

\begin{itemize}
\tightlist
\item
  Age
\item
  Sex
\item
  Systolic blood pressure
\item
  Glasgow Coma Scale
\item
  Injury Severity Score
\item
  Mechanism of injury
\end{itemize}

These are known individual-level prognostic factors for the primary
outcome. These covariates will be included in the models specified in
Equation~\ref{eq-logit-model} and Equation~\ref{eq-identity-model} as
fixed effects.

\hypertarget{subgroup-analyses}{%
\subsubsection{Subgroup analyses}\label{subgroup-analyses}}

We will perform the following subgroup analyses:

\begin{itemize}
\tightlist
\item
  geographical region, defined using the state in which the
  participating hospital is located. Demonstrating the consistency of
  any effect across multiple regions will enhance the generalisibility
  of the results;
\item
  age groups, defined as older adolescents (15-19 years), young adults
  (20-24 years), adults (25-59 years), and older adults (60 years and
  older) {[}\textsuperscript{6};
\item
  sex, using the levels male and female;
\item
  clinical cohorts, defined as blunt multisystem trauma, penetrating
  trauma, and severe isolated traumatic brain injury, with modification
  to avoid overlap between the cohorts; and
\item
  cluster size.
\end{itemize}

These subgroup analyses will be conducted by adding the subgroup
variable and the interaction between the subgroup variable and the
intervention exposure variable as fixed effects to the models specified
in Equation~\ref{eq-logit-model} and Equation~\ref{eq-identity-model}.

\hypertarget{treatment-of-missing-data}{%
\subsubsection{Treatment of missing
data}\label{treatment-of-missing-data}}

We will present the frequency and percentage of missing data for all
variables. If the percentage of missing data for the primary outcome is
less than 10\% then we will perform a complete case analysis. If the
percentage of missing data for the primary outcome is 10\% or more, then
we will handle missing data using multiple imputation by chained
equations (MICE), imputing data for the primary outcome as well as all
covariates included in the fully adjusted model. The number of
imputations will be determined by the percentage of missing data, with a
minimum of 20 imputations.

\hypertarget{analysis-of-secondary-outcomes}{%
\subsection{Analysis of secondary
outcomes}\label{analysis-of-secondary-outcomes}}

\hypertarget{all-cause-mortality-within-24-hours-30-days-and-three-months-of-arrival-at-the-emergency-department}{%
\subsubsection{All cause mortality within 24 hours, 30 days and three
months of arrival at the emergency
department}\label{all-cause-mortality-within-24-hours-30-days-and-three-months-of-arrival-at-the-emergency-department}}

We will use the model with the logit link specified in
Equation~\ref{eq-logit-model} and the model with the identity link
specified in Equation~\ref{eq-identity-model} to estimate the OR and
risk difference for these mortality outcomes.

\hypertarget{quality-of-life-within-seven-days-of-discharge-and-at-30-days-and-three-months-of-arrival-at-the-emergency-department}{%
\subsubsection{Quality of life within seven days of discharge, and at 30
days and three months of arrival at the emergency
department}\label{quality-of-life-within-seven-days-of-discharge-and-at-30-days-and-three-months-of-arrival-at-the-emergency-department}}

Quality of life will be measured by the official and validated
translations of the EQ5D5L. This tool assesses five dimensions of
health-related quality of life: mobility, self-care, usual activities,
pain/discomfort, and anxiety/depression. Each dimension is rated on a
likert scale from 1 to 5. There is also a visual analogue scale (VAS)
for self-rated quality of life, ranging from 0 to 100. For each of the
five dimensions we will use a mixed effects ordinal model as specified
in Equation~\ref{eq-ordinal-model}.

\begin{equation}\protect\hypertarget{eq-ordinal-model}{}{
\text{logit}(\text{Pr}(Y_{bkti} \leq j)) = \mu_j + \beta_{bt} + \theta X_{bkt} + \alpha_{bk} + \gamma_{bkt}
}\label{eq-ordinal-model}\end{equation}

Where:

\begin{itemize}
\tightlist
\item
  \(\mu_j\) is the intercept for the \(j\)-th category of the EQ5D
  dimension (\(j\) = 1, 2, 3, 4, 5).
\end{itemize}

The VAS will be analysed using a linear mixed effects model as specified
in Equation~\ref{eq-linear-model}.

\begin{equation}\protect\hypertarget{eq-linear-model}{}{
\text{VAS}_{bkti} = \mu + \beta_{bt} + \theta X_{bkt} + \alpha_{bk} + \gamma_{bkt} + \epsilon_{bkti}, \quad \epsilon_{bkti} \sim N(0, \sigma^2)
}\label{eq-linear-model}\end{equation}

Where:

\begin{itemize}
\tightlist
\item
  \(\epsilon_{bkti}\) is the error term for patient \(i\) in cluster
  \(k\) in period \(t\) in batch \(b\), assumed to be normally
  distributed with mean 0 and variance \(\sigma^2\).
\end{itemize}

\hypertarget{disability-within-seven-days-of-discharge-and-at-30-days-and-three-months-of-arrival-at-the-emergency-department}{%
\subsubsection{Disability within seven days of discharge, and at 30 days
and three months of arrival at the emergency
department}\label{disability-within-seven-days-of-discharge-and-at-30-days-and-three-months-of-arrival-at-the-emergency-department}}

We will measure disability using the WHO Disability Assessment Schedule
2.0 (WHODAS 2.0)\textsuperscript{7}. This tool assesses six domains of
functioning: cognition, mobility, self-care, getting along, life
activities, and participation. Each domain is rated on a likert scale
from 1 to 5, with 1 indicating no difficulties and 5 indicating extreme
difficulties. We will analyse each domain separately using a mixed
effects ordinal model as specified in Equation~\ref{eq-ordinal-model}.
We will also calculate a WHODAS 2.0 summary score using the method
referred to as the ``complex scoring'' method. This method involves
summing the item scores within each of the six domains, then summing the
scores of all domains, and finally transforming the total score to a
0-100 scale. We will analyse the summary score using a linear mixed
effects model as specified in Equation~\ref{eq-linear-model}.

\hypertarget{return-to-work-at-30-days-and-three-months-after-arrival-at-the-emergency-department}{%
\subsubsection{Return to work at 30 days and three months after arrival
at the emergency
department}\label{return-to-work-at-30-days-and-three-months-after-arrival-at-the-emergency-department}}

We will analyse return to work as a dichotomous variable using a mixed
effects binomial model with a logit link as specified in
Equation~\ref{eq-logit-model}.

\hypertarget{length-of-emergency-department-stay}{%
\subsubsection{Length of emergency department
stay}\label{length-of-emergency-department-stay}}

We will analyse length of emergency department stay as a continuous
variable using a linear mixed effects model as specified in
Equation~\ref{eq-linear-model}.

\hypertarget{length-of-hospital-stay}{%
\subsubsection{Length of hospital stay}\label{length-of-hospital-stay}}

We will analyse length of hospital stay as a continuous variable using a
linear mixed effects model as specified in
Equation~\ref{eq-linear-model}.

\hypertarget{intensive-care-unit-admission}{%
\subsubsection{Intensive care unit
admission}\label{intensive-care-unit-admission}}

We will analyse intensive care unit admission as a dichotomous variable
using a mixed effects binomial model with a logit link as specified in
Equation~\ref{eq-logit-model}.

\hypertarget{length-of-intensive-care-unit-stay}{%
\subsubsection{Length of intensive care unit
stay}\label{length-of-intensive-care-unit-stay}}

We will analyse length of intensive care unit stay as a continuous
variable using a linear mixed effects model as specified in
Equation~\ref{eq-linear-model}.

\hypertarget{references}{%
\section*{References}\label{references}}
\addcontentsline{toc}{section}{References}

\hypertarget{refs}{}
\begin{CSLReferences}{0}{0}
\leavevmode\vadjust pre{\hypertarget{ref-Kasza2022}{}}%
\CSLLeftMargin{1. }%
\CSLRightInline{Kasza, J. \emph{et al.} The batched stepped wedge
design: A design robust to delays in cluster recruitment. \emph{Stat
Med} \textbf{41}, 3627--3641 (2022).}

\leavevmode\vadjust pre{\hypertarget{ref-R}{}}%
\CSLLeftMargin{2. }%
\CSLRightInline{R Core Team. \emph{R: A language and environment for
statistical computing}. (R Foundation for Statistical Computing, 2023).}

\leavevmode\vadjust pre{\hypertarget{ref-Hemming2018}{}}%
\CSLLeftMargin{3. }%
\CSLRightInline{Hemming, K. \emph{et al.} Reporting of stepped wedge
cluster randomised trials: Extension of the CONSORT 2010 statement with
explanation and elaboration. \emph{BMJ} k1614 (2018).}

\leavevmode\vadjust pre{\hypertarget{ref-kenward_small_1997}{}}%
\CSLLeftMargin{4. }%
\CSLRightInline{Kenward, M. G. \emph{et al.} Small {Sample} {Inference}
for {Fixed} {Effects} from {Restricted} {Maximum} {Likelihood}.
\emph{Biometrics} \textbf{53}, 983--997 (1997).}

\leavevmode\vadjust pre{\hypertarget{ref-kenny_analysis_2022}{}}%
\CSLLeftMargin{5. }%
\CSLRightInline{Kenny, A. \emph{et al.} Analysis of stepped wedge
cluster randomized trials in the presence of a time-varying treatment
effect. \emph{Statistics in medicine} 10.1002/sim.9511 (2022).}

\leavevmode\vadjust pre{\hypertarget{ref-Diaz2021}{}}%
\CSLLeftMargin{6. }%
\CSLRightInline{Diaz, T. \emph{et al.} A call for standardised
age-disaggregated health data. \emph{The Lancet Healthy Longevity}
\textbf{2}, e436--e443 (2021).}

\leavevmode\vadjust pre{\hypertarget{ref-ustun_measuring_2010}{}}%
\CSLLeftMargin{7. }%
\CSLRightInline{Ustun, T. B. \emph{et al.} \emph{Measuring {Health} and
{Disability}: {Manual} for {WHO} {Disability} {Assessment} {Schedule}
({WHODAS} 2.0)}. (World Health Organization, 2010).}

\end{CSLReferences}



\end{document}
